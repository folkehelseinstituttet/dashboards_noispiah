\documentclass[a4paper,12pt]{article}

\usepackage[top=3cm, bottom=2.5cm, left=1.25cm, right=1.25cm, headsep=1cm]{geometry}
\renewcommand{\baselinestretch}{1.5} 

\usepackage{fontspec}
\setmainfont{Ubuntu}

\let\oldtabular\tabular
\renewcommand{\tabular}{\tiny\oldtabular}

\usepackage{svg}

\usepackage{fancyhdr}
\pagestyle{fancy}

\fancyhf{}
\fancyfoot[R]{\sffamily\nouppercase{\thepage}}
\fancyfoot[L]{\sffamily\nouppercase{Folkehelseinstituttet, Oslo, Norge}}
\fancyhead[L]{\sffamily\nouppercase{Prevalensundersøkelsen for helsetjenesteassosierte infeksjoner og
antibiotikabruk \newline Fræna --- Høst 2017}}
\renewcommand{\headrulewidth}{0.5pt}
\renewcommand{\footrulewidth}{0.5pt}

\usepackage{pdflscape}
\usepackage{blindtext}

\begin{document}

\begin{titlepage}
\begin{figure}
    \centering
    \def\svgwidth{0.3\columnwidth}
    \input{fhi.pdf_tex}
\end{figure}

  \vspace*{9em}{\raggedright\Huge
    \textbf{Prevalensundersøkelsen for helsetjenesteassosierte infeksjoner og
antibiotikabruk}\par
  }
  \vspace{2em}{\raggedright
    Fræna --- Høst 2017
  }

  \vspace*{\fill}{\raggedleft\vfill{
        Torunn Alberg \\
        Hanne-Merete Eriksen \\
        Hege Line Magnussen Løwer \\
        Richard Aubrey White \\
      }\par}\par
\end{titlepage}

\newpage 

\section{Deltagelse, forekomst av helsetjenesteassosierte infeksjoner og
bruk av antibiotika i
sykehjem}\label{deltagelse-forekomst-av-helsetjenesteassosierte-infeksjoner-og-bruk-av-antibiotika-i-sykehjem}

\blindtext

\begin{landscape}

\begin{table}[ht]
\centering
\caption{Deltagelse, forekomst av helsetjenesteassosierte infeksjoner og bruk av antibiotika i sykehjem} 
\begin{tabular}{|r||r|r|r|r||r|r|r|r|}
  \hline
  & \multicolumn{4}{c||}{HAI} & \multicolumn{4}{c|}{Bruk av antibiotika} \\
  \hline
 \multicolumn{1}{|r||}{Fylke} & \multicolumn{1}{p{1.0cm}}{Antall sykehjem} & \multicolumn{1}{p{1.1cm}}{Antall beboere} & \multicolumn{1}{p{1.5cm}}{Andel beboere med HAI} & \multicolumn{1}{p{1.6cm}||}{Prevalens av HAI i prosent} & \multicolumn{1}{p{1.0cm}}{Antall sykehjem} & \multicolumn{1}{p{1.1cm}}{Antall beboere} & \multicolumn{1}{p{2cm}}{Andel beboere som fikk antibiotika} & \multicolumn{1}{p{2.75cm}|}{Prevalens av antibiotikabruk i prosent}  \\
 \hline
Akershus & 29 & 1578 & 4.6 & 4.8 & 30 & 1718 & 4.9 & 5.0 \\ 
  Aust-Agder & 14 &  427 & 3.0 & 3.0 & 16 &  486 & 7.6 & 7.2 \\ 
  Buskerud & 33 & 1417 & 5.2 & 5.4 & 34 & 1446 & 6.6 & 4.6 \\ 
  Finnmark & 19 &  435 & 6.2 & 6.7 & 20 &  476 & 9.9 & 9.5 \\ 
  Hedmark & 29 & 1360 & 4.1 & 4.6 & 32 & 1409 & 6.5 & 6.5 \\ 
  Hordaland & 58 & 3270 & 3.6 & 3.8 & 60 & 3292 & 5.4 & 5.3 \\ 
  Møre og Romsdal & 33 & 1164 & 6.4 & 7.2 & 35 & 1195 & 9.1 & 6.9 \\ 
  Nordland & 45 & 1367 & 5.1 & 5.7 & 46 & 1420 & 7.0 & 6.2 \\ 
  Oppland & 26 & 1152 & 4.3 & 4.9 & 29 & 1255 & 7.2 & 6.6 \\ 
  Oslo & 40 & 3564 & 4.0 & 4.2 & 41 & 3590 & 4.1 & 3.8 \\ 
  Rogaland & 46 & 1975 & 5.5 & 5.6 & 47 & 2001 & 8.9 & 8.8 \\ 
  Sogn og Fjordane & 22 &  706 & 3.1 & 3.3 & 23 &  770 & 4.2 & 4.3 \\ 
  Telemark & 11 &  401 & 4.5 & 4.7 & 12 &  424 & 7.1 & 7.5 \\ 
  Troms & 32 & 1043 & 4.9 & 5.5 & 31 &  971 & 9.4 & 8.7 \\ 
  Trøndelag & 67 & 2684 & 4.7 & 5.0 & 67 & 2658 & 7.7 & 7.3 \\ 
  Vest-Agder & 20 &  784 & 5.0 & 5.2 & 22 &  850 & 7.9 & 8.0 \\ 
  Vestfold & 26 & 1261 & 5.8 & 6.0 & 27 & 1281 & 8.2 & 6.9 \\ 
  Østfold & 34 & 1672 & 4.8 & 5.0 & 35 & 1748 & 8.1 & 8.2 \\ 
   \hline
\end{tabular}
\end{table}

\end{landscape}

\newpage

\section{Prevalens av helsetjenesteassosierte infeksjoner blant beboere
i sykehjem per
avdelingstype}\label{prevalens-av-helsetjenesteassosierte-infeksjoner-blant-beboere-i-sykehjem-per-avdelingstype}

\blindtext

hello

hi

\includegraphics{sykehjem_files/figure-latex/unnamed-chunk-3-1.pdf}

\newpage

\section{Andel forskrivninger av antibiotika til forebygging og
behandling per
indikasjon}\label{andel-forskrivninger-av-antibiotika-til-forebygging-og-behandling-per-indikasjon}

\blindtext

\newpage

\section{Forskrivning av antibiotika (virkestoff) per
indikasjon}\label{forskrivning-av-antibiotika-virkestoff-per-indikasjon}

\blindtext

\end{document}